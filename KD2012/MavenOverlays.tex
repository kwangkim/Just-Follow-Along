\documentclass[xcolor=dvipsnames,14pt]{beamer}
\usetheme{kualidays2012}
\usecolortheme[RGB={125,25,25}]{structure}
\setbeamerfont{structure}{family=\rmfamily,series=\bfseries} 
\setbeamerfont{subtitle}{family=\sffamily,series=\bfseries} 
\setbeamercolor{normal text}{bg=brown!46}

\begin{document}

\title{Si Vis Bellum Para Bellum}
\subtitle{KC Customization using Maven's WAR overlays}
\author[Leo]{Leo Przybylski}

\institute[rSmart]{rSmart\inst{1} \\[1ex] 
  \texttt{leo@rsmart.com}
}

\begin{frame}[plain]
  \titlepage
\end{frame}

\begin{frame}{Overview}
\end{frame}

\begin{frame}{About Presenters}
\end{frame}

\begin{frame}{How Overlays Work}
  \begin{itemize}
    \item A fileset or maven project copied over a webapp.
    \item Webapp does not have to be a maven project.
    \item The overlay does not have to be a complete webapp.
    \item An overlay does not even have to be very different at all.
  \end{itemize}
\end{frame}

\begin{frame}{Overlay Patterns}
  \begin{itemize}
    \item Overlaying configuration modifications.
    \item Overlay to activate a module.
    \item Overlay theming for UX/UI.
    \item 
  \end{itemize}
\end{frame}

\begin{frame}{Advantages of Overlays}
  \begin{itemize}
    \item Offers flexibility in your project by allowing granular
      modifications at the project level.
    \item Smaller projects for tracking changes with.
    \item Easier to modularize your project.
    \item Better code reuse.
  \end{itemize}
\end{frame}

\begin{frame}{Overwriting Changes in an Overlaid Project}
\end{frame}

\end{document}