\addcontentsline{toc}{part}{Exercise 4: Create Maintenance Document}
\part*{Exercise 4: \\
Setup Business Object}

\addcontentsline{toc}{section}{Exercise 5: Create Maintenance Document}
{\setlength{\baselineskip}%
  {0.0\baselineskip}
  \section*{\flushright Exercise r: \\
Setup Business Object}
  \hrulefill \par}

\addcontentsline{toc}{subsection}{Description}
\subsection*{Description}

\addcontentsline{toc}{subsection}{Goals}
\subsection*{Goals}

\addcontentsline{toc}{subsection}{Instructions}
\subsection*{Instructions}

\subsubsection*{1 Create Business Object Class}
Create a file in \textbf{work/src/org/kuali/kfs/sys/businessobject/}
called \textbf{FerpaCertification}
\begin{minted}{java}
/*
 * Copyright 2005-2008 The Kuali Foundation
 * 
 * Licensed under the Educational Community License, Version 2.0 (the "License");
 * you may not use this file except in compliance with the License.
 * You may obtain a copy of the License at
 * 
 * http://www.opensource.org/licenses/ecl2.php
 * 
 * Unless required by applicable law or agreed to in writing, software
 * distributed under the License is distributed on an "AS IS" BASIS,
 * WITHOUT WARRANTIES OR CONDITIONS OF ANY KIND, either express or implied.
 * See the License for the specific language governing permissions and
 * limitations under the License.
 */
package org.kuali.kfs.sys.businessobject;


/**
 * Ferpa Certification Business Object
 */
public class FerpaCertification extends PersistableBusinessObjectBase implements Inactivateable {
  
  private Long id;
  private String principalId;
  private String Person;

  /**
  * Default no-arg constructor.
  */
  
  public FerpaCertification() {
        super();
    }

    /**
     * Gets the id attribute.
     * 
     * @return Returns the id.
     */
    public String getId() {
        return id;
    }


    /**
     * Sets the id attribute value.
     * 
     * @param id The id to set.
     */
    public void setId(Long id) {
        this.id = id;
    }


    /**
     * Gets the principalId attribute.
     * 
     * @return Returns the principalId.
     */
    public Long getPrincipalId() {
        return principalId;
    }


    /**
     * Sets the principalId attribute value.
     * 
     * @param principalId The principalId to set.
     */
    public void setPrincipalId(String principalId) {
        this.principalId = principalId;
    }


    /**
     * Gets the person attribute.
     * 
     * @return Returns the person.
     */
    public String getPerson() {
        return person;
    }


    /**
     * Sets the person attribute value.
     * 
     * @param person The person to set.
     */
    public void setPerson(String person) {
        this.person = person;
    }
}
\end{minted}

\subsubsection*{2 Create Business DataDictionary Metadata}
\begin{minted}{xml}
<?xml version="1.0" encoding="UTF-8"?><beans xmlns="http://www.springframework.org/schema/beans" xmlns:xsi="http://www.w3.org/2001/XMLSchema-instance" xmlns:p="http://www.springframework.org/schema/p" xsi:schemaLocation="http://www.springframework.org/schema/beans         http://www.springframework.org/schema/beans/spring-beans-2.0.xsd">
<!--
 Copyright 2008-2009 The Kuali Foundation
 
 Licensed under the Educational Community License, Version 2.0 (the "License");
 you may not use this file except in compliance with the License.
 You may obtain a copy of the License at
 
 http://www.opensource.org/licenses/ecl2.php
 
 Unless required by applicable law or agreed to in writing, software
 distributed under the License is distributed on an "AS IS" BASIS,
 WITHOUT WARRANTIES OR CONDITIONS OF ANY KIND, either express or implied.
 See the License for the specific language governing permissions and
 limitations under the License.
-->

  <bean id="FerpaCertification" parent="FerpaCertification-parentBean"/>

  <bean id="FerpaCertification-parentBean" abstract="true" parent="BusinessObjectEntry">
    <property name="businessObjectClass" value="org.kuali.kfs.sys.businessobject.FerpaCertification"/>
    <property name="inquiryDefinition">
      <ref bean="FerpaCertification-inquiryDefinition"/>
    </property>
    <property name="lookupDefinition">
      <ref bean="FerpaCertification-lookupDefinition"/>
    </property>
    <property name="titleAttribute" value="bankCode"/>
    <property name="objectLabel" value="FerpaCertification"/>
    <property name="attributes">
      <list>
        <ref bean="FerpaCertification-bankCode"/>
        <ref bean="FerpaCertification-bankName"/>
        <ref bean="FerpaCertification-bankShortName"/>
        <ref bean="FerpaCertification-bankRoutingNumber"/>
        <ref bean="FerpaCertification-bankAccountNumber"/>
        <ref bean="FerpaCertification-bankAccountDescription"/>
        <ref bean="FerpaCertification-cashOffsetFinancialChartOfAccountCode"/>
        <ref bean="FerpaCertification-cashOffsetAccountNumber"/>
        <ref bean="FerpaCertification-cashOffsetSubAccountNumber"/>
        <ref bean="FerpaCertification-cashOffsetObjectCode"/>
        <ref bean="FerpaCertification-cashOffsetSubObjectCode"/>
        <ref bean="FerpaCertification-continuationFerpaCertificationCode"/>
        <ref bean="FerpaCertification-bankDepositIndicator"/>
        <ref bean="FerpaCertification-bankDisbursementIndicator"/>
        <ref bean="FerpaCertification-bankAchIndicator"/>
        <ref bean="FerpaCertification-bankCheckIndicator"/>
        <ref bean="FerpaCertification-active"/>
      </list>
    </property>
  </bean>

<!-- Attribute Definitions -->


  <bean id="FerpaCertification-id" parent="FerpaCertification-id-parentBean"/>

  <bean id="FerpaCertification-id-parentBean" abstract="true" parent="AttributeDefinition">
    <property name="name" value="id"/>
    <property name="forceUppercase" value="true"/>
    <property name="label" value="FerpaCertification Code"/>
    <property name="shortLabel" value="FerpaCertification Code"/>
    <property name="maxLength" value="4"/>
    <property name="validationPattern">
      <ref bean="AlphaNumericValidation" />
    </property>
    <property name="required" value="true"/>
    <property name="control">
      <ref bean="HiddenControl" />
    </property>
  </bean>
  <bean id="FerpaCertification-principalId" parent="FerpaCertification-principalId-parentBean"/>

  <bean id="FerpaCertification-principalId-parentBean" abstract="true" parent="PersonImpl-principalId">
    <property name="name" value="principalId"/>
  </bean>

  <bean id="FerpaCertification-person.principalName" parent="FerpaCertification-person.principalName-parentBean"/>

  <bean id="FerpaCertification-person.principalName-parentBean" abstract="true" parent="PersonImpl-principalName">
    <property name="name" value="person.principalName"/>
    <property name="control">
      <bean parent="KualiUserControlDefinition" p:personNameAttributeName="person.name" p:universalIdAttributeName="principalId" p:userIdAttributeName="person.principalName"/>
    </property>
    <property name="required" value="false"/>
  </bean>

  <bean id="FerpaCertification-active" parent="FerpaCertification-active-parentBean"/>      
  <bean id="FerpaCertification-active-parentBean" abstract="true" parent="GenericAttributes-activeIndicator">
    <property name="name" value="active"/>
    <property name="required" value="false"/>
  </bean>
  
<!-- Business Object Inquiry Definition -->


  <bean id="FerpaCertification-inquiryDefinition" parent="FerpaCertification-inquiryDefinition-parentBean"/>

  <bean id="FerpaCertification-inquiryDefinition-parentBean" abstract="true" parent="InquiryDefinition">
    <property name="title" value="FerpaCertification Inquiry"/>
    <property name="inquirySections">
      <list>
        <bean parent="InquirySectionDefinition">
          <property name="title" value=""/>
          <property name="numberOfColumns" value="1"/>
          <property name="inquiryFields">
            <list>
              <bean parent="FieldDefinition" p:attributeName="id"/>
              <bean parent="FieldDefinition" p:attributeName="principalId"/>
              <bean parent="FieldDefinition" p:attributeName="active"/>
            </list>
          </property>
        </bean>
      </list>
    </property>
  </bean>
  
<!-- Business Object Lookup Definition -->


  <bean id="FerpaCertification-lookupDefinition" parent="FerpaCertification-lookupDefinition-parentBean"/>

  <bean id="FerpaCertification-lookupDefinition-parentBean" abstract="true" parent="LookupDefinition">
    <property name="title" value="FerpaCertification Lookup"/>
    
    <property name="defaultSort">
      <bean parent="SortDefinition">
        <property name="attributeNames">
          <list>
            <value>bankCode</value>
          </list>
        </property>
      </bean>
    </property>
    <property name="lookupFields">
      <list>
              <bean parent="FieldDefinition" p:attributeName="principalId"/>
              <bean parent="FieldDefinition" p:attributeName="active"/>
      </list>
    </property>
    <property name="resultFields">
      <list>
              <bean parent="FieldDefinition" p:attributeName="person.principalName"/>
              <bean parent="FieldDefinition" p:attributeName="active"/>
      </list>
    </property>
  </bean>
</beans>\end{minted}

\subsubsection*{3 Create O/R Mapping}

\begin{minted}{xml}
\end{minted}

\subsubsection*{Add a link to the portal}

\newpage
{\setlength{\baselineskip}%
  {0.0\baselineskip}
  \section*{Notes}
  \hrulefill \par}