\documentclass[12pt,notitlepage]{article}
\author{Leo Przybylski \\
\texttt{przybyls@arizona.edu}}
\usepackage{listings}
\usepackage{color}
\usepackage{graphicx}

\title{Integrating LDAP and KIM }


\begin{document}
\maketitle

\abstract{
Kuali Identity Management (KIM) features a reference implementation that requires an RDBMS datastore. Most schools, however, need to integrate their Kuali installation with an external system that is the master datasource for identity information and authentication. Since KIM is the service that all Kuali applications talk to, it is important that it be easy to customize it to talk to external systems.

This session covers the ease of integrating Spring LDAP with KIM and the caveats of KIM integration and alternative datastores. At the time of writing this, Rice does not have a reference KIM LDAP implementation. This session will explore the integrating a software extension within a KFS or Rice application.

Attendees will take away:
\begin{enumerate}
\item Experience taking an existing Kuali application and adding a directory service datastore for entities
\item Learn about what goes into mapping a directory service to KIM
\item Understand how to implement the necessary service overrides
\end{enumerate}

\end{document}
